%%%%%%%%%%%%%%%%%%%%%%%%%%%%%%%%%%%%%%%%%%%%%%%%%%%%%%%%%%%%%%%%%
% Contents: Who contributed to this Document
% $Id: overview.tex 456 2011-04-06 09:10:27Z oetiker $
%%%%%%%%%%%%%%%%%%%%%%%%%%%%%%%%%%%%%%%%%%%%%%%%%%%%%%%%%%%%%%%%%

% Because this introduction is the reader's first impression, I have
% edited very heavily to try to clarify and economize the language.
% I hope you do not mind! I always try to ask "is this word needed?"
% in my own writing but I don't want to impose my style on you... 
% but here I think it may be more important than the rest of the book.
% --baron

%\chapter{Preface}
\chapter{서문}

%\LaTeX{} \cite{manual} is a typesetting system that is very 
%suitable for producing scientific and mathematical documents of high
%typographical quality. It is also suitable for producing all
%sorts of other documents, from simple letters to complete books.
%\LaTeX{} uses \TeX{} \cite{texbook} as its formatting engine.
\LaTeX\medspace\cite{manual}은 과학 및 수학 문서를 작성하는 데 적당한 조판 시스템으로
대단히 뛰어난 조판 품질을 얻을 수 있게 한다.
또한 단순한 편지에서 완전한 단행본에 이르기까지 여러 종류의 문서를 만드는 데도 적합하다.
\LaTeX{}은 \TeX{} \cite{texbook}을 조판 엔진으로 사용한다.

%This short introduction describes \LaTeXe{} and should be sufficient
%for most applications of \LaTeX. Refer to~\cite{manual,companion} for
%a complete description of the \LaTeX{} system.
이 짧지 않은\footnote{원문에는 short으로 되어 있으며 책 제목에는 not so short로 되어 있음.
왜 다른지 원저자에게 문의하여 다음에는 업데이트하겠다는 메일을 받음. --옮긴이} 입문서는 \LaTeXe{}에 대해 설명하며 \LaTeX{}의 거의 모든 응용에 충분하다.
\LaTeX{} 시스템에 대한 완전한 설명을 보려면 \cite{manual,companion}\을
참조하시오.

\bigskip
%\noindent This introduction is split into 6 chapters:
\noindent 이 입문서는 여섯 장으로 나뉘어 있다.
\begin{description}
%\item[Chapter 1] tells you about the basic structure of \LaTeXe{}
%  documents. You will also learn a bit about the history of \LaTeX{}.
%  After reading this chapter, you should have a rough understanding how
%  \LaTeX{} works.
\item[제 1 장] \LaTeXe{} 문서의 기본 구조에 대해 말한다.
 \LaTeX{}의 역사에 대해서도 조금 얘기한다. 
  이 장을 읽으면 \LaTeX 이 어떻게 동작하는가에 대해 대강 이해할 수 있게 될 것이다.

%\item[Chapter 2] goes into the details of typesetting your
%  documents. It explains most of the essential \LaTeX{} commands and
%  environments. After reading this chapter, you will be able to write
%  your first documents.
\item[제 2 장] 문서 조판의 세부사항을 다룬다.
대부분의 본질적인 \LaTeX{} 명령과 환경을 설명한다.
이 장을 읽으면, 첫번째 문서를 쓸 수 있게 될 것이다. 
 
%\item[Chapter 3] explains how to typeset formulae with \LaTeX. Many
%  examples demonstrate how to use one of \LaTeX{}'s
%  main strengths. At the end of the chapter are tables listing
%  all mathematical symbols available in \LaTeX{}.
\item[제 3 장] 수학식을 \LaTeX 으로  식자하는 방법을 설명한다.
\LaTeX 의 가장 강력한 기능 중 하나인 수식 표현을 많은 예제를 통하여 보여준다.
이 장의 끝에는 \LaTeX 으로 표현할 수 있는 모든 수학 기호를 표로 정리했다.

%\item[Chapter 4] explains indexes,  bibliography generation and
%  inclusion of EPS graphics. It introduces creation of PDF documents with pdf\LaTeX{}
%  and presents some handy extension packages.
\item[제 4 장] 색인, 문헌목록 생성과 EPS 그림 삽입을 설명한다.
pdf\LaTeX 을 이용하여 PDF 문서를 만드는 것에 대해 소개하고  몇가지 간편한 확장 패키지를 제공한다.

%\item[Chapter 5] shows how to use \LaTeX{} for creating graphics. Instead
% of drawing a picture with some graphics program, saving it to a file and
% then including it into \LaTeX{}, you describe the picture and have \LaTeX{}
% draw it for you.
\item[제 5 장] \LaTeX 으로 그림을 만드는 방법을 다룬다.
그래픽 프로그램으로 그림을 그려서 파일로 저장하여 \LaTeX 에 넣는 대신, \LaTeX 로 그림을 표현하는
방법을 설명한다.

%\item[Chapter 6] contains some potentially dangerous information about
%  how to alter the
%  standard document layout produced by \LaTeX{}. It will tell you how  to
%  change things such that the beautiful output of \LaTeX{}
%  turns ugly or stunning, depending on your abilities.
\item[제 6 장] \LaTeX 이 만들어내는 표준 문서 레이아웃을 변경하는,
약간 위험할 수도 있는 내용을 포함하고 있다. \LaTeX 의 아름다운 출력물을 어떻게 하면 망치거나 (능력에 따라서) 근사하게 바꿀 수 있는가를 알려준다.
\end{description}
\bigskip
%\noindent It is important to read the chapters in order---the book is
%not that big, after all. Be sure to carefully read the examples,
%because a lot of the information is in the
%examples placed throughout the book.
\noindent 각 장을 순서대로 읽는 것이 중요하다. 이 책은 분량이 그다지 많지 않다. 특히
예제를 주의깊게 보아야 한다. 이 책 전체에 걸쳐 나타나는 예제들에 중요한 정보가 담겨 있기
때문이다. 

\bigskip
%\noindent \LaTeX{} is available for most computers, from the PC and Mac to large
%UNIX and VMS systems. On many university computer clusters you will
%find that a \LaTeX{} installation is available, ready to use.
%Information on how to access
%the local \LaTeX{} installation should be provided in the \guide. If
%you have problems getting started, ask the person who gave you this
%booklet. The scope of this document is \emph{not} to tell you how to
%install and set up a \LaTeX{} system, but to teach you how to write
%your documents so that they can be processed by~\LaTeX{}.
\noindent \LaTeX 은 PC나 매킨토시에서 대규모 UNIX나 VMS에 이르기까지 대부분의
컴퓨터에서 실행된다. 대학의 컴퓨터실에는 대개 \LaTeX 이 이미 설치되어 있을 것이다. 현재
자신이 이용하는 시스템에 \LaTeX 이 설치되어 있는지, 어떻게 사용하여야 하는지에 대해서 알고
싶으면 \guide 를 참고한다. 아무리 해도 잘 되지 않으면 이 책자를 읽으라고 권한 사람에게
물어보라.  이 책이 다루는 범위는 \LaTeX 으로 문서를 작성하는 방법에 대해서 알려주고자 하는
것이지, \LaTeX\ 시스템을 설치하고 설정하는 데 대한 것이 아니다.

\bigskip
%\noindent If you need to get hold of any \LaTeX{} related material, 
%have a look at one of the Comprehensive \TeX{} Archive Network
%(CTAN) sites. The homepage is at
%\url{http://www.ctan.org}.
\noindent \LaTeX{} 관련 자료가 필요하면, 
Comprehensive \TeX{} Archive Network (CTAN) 사이트를 보라. 홈페이지는
\url{http://www.ctan.org}.

%You will find other references to CTAN throughout the book, especially
%pointers to software and documents you might want to download. Instead
%of writing down complete urls, I just wrote \texttt{CTAN:} followed by
%whatever location within the CTAN tree you should go to.
이 책에 CTAN에 대한 참조가 여러 번 나온다. 특히 다운로드하고자 하는 소프트웨어나 문서를
가리킬 때 그러하다. 완전한 URL을 기록하는 대신 \texttt{CTAN:}이라고 쓰고 그 다음에
CTAN 트리 상의 위치를 썼다.

%If you want to run \LaTeX{} on your own computer, take a look at what
%is available from \CTAN|systems|.
\LaTeX{}을 컴퓨터에서 실행하려면,  \CTAN|systems| 에서 무엇이 가능한지를 보라.

\vspace{\stretch{1}}
%\noindent If you have ideas for something to be
%added, removed or altered in this document, please let me know. I am
%especially interested in feedback from \LaTeX{} novices about which
%bits of this intro are easy to understand and which could be explained
%better.
\noindent 이 문서에 더하거나 빼거나 바꿀  부분에 대한 의견이 있다면
저자에게 알려주기 바란다. 저자는 이 안내서의 내용이 이해하기 쉬운지 더 좋은 설명 방법은 없겠는지에
대한 \LaTeX\ 초보자로부터의 피드백에 특히 관심이 있다.

\bigskip
\begin{verse}
\contrib{Tobias Oetiker}{tobi@oetiker.ch}%
\noindent{OETIKER+PARTNER AG\\Aarweg 15\\4600 Olten\\Switzerland}
\end{verse}
\vspace{\stretch{1}}
%\noindent The current version of this document is available on\\
%\CTAN|info/lshort|
\noindent 이 문서의 현재 버전은 아래에서 얻을 수 있다.\\
\CTAN|info/lshort|

\endinput



%

% Local Variables:
% TeX-master: "lshort2e"
% mode: latex
% mode: flyspell
% End:
