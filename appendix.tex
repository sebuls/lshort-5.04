\appendix
%\chapter{Installing \LaTeX}
\chapter{\LaTeX{} 설치}
\begin{intro}
%Knuth published the source to \TeX{} back in a time when nobody knew
%about OpenSource and/or Free Software. The License that comes with \TeX{}
%lets you do whatever you want with the source, but you can only call the
%result of your work \TeX{} if the program passes a set of tests Knuth has
%also provided. This has lead to a situation where we have free \TeX{}
%implementations for almost every Operating System under the sun. This chapter
%will give some hints on what to install on Linux, Mac OS X and Windows, to
%get a working \TeX{} setup.
오픈소스와 자유소프트웨어를 아무도 모르고 있을 때 Knuth는 소스를 \TeX{}로  출판했다.
\TeX{}과 함께 오는 저작권은 소스를 가지고 원하는 것을 할수 있게 하지만,
Knuth가 제공한 테스트를 프로그램이 통과했을 때만 \TeX{}작업의 결과를 부를 수 있다.
온세상의 거의 모든 운영체제에 자유로운 \TeX{}구현을 갖고 있을 때 이 상황을 이어갈 수 있다.
이 장은 \TeX{} 설정작업을 하기 위해 리눅스, 맥OS와 윈도에 무엇을 설치할지 힌트를 줄 것이다.
\end{intro}

%\section{What to Install}
\section{설치할 프로그램}
%To use \LaTeX{} on any computer system, you need several programs.
\LaTeX{}을 컴퓨터에서 쓰려면, 여러 프로그램이 필요하다.
\begin{enumerate}

%\item The \TeX{}/\LaTeX{} program for processing your \LaTeX{} source files
%into typeset PDF or DVI documents.
\item \TeX{}/\LaTeX{} program for processing your \LaTeX{} source files
into typeset PDF or DVI documents.

%\item A text editor for editing your \LaTeX{} source files. Some products even let
%you start the \LaTeX{} program from within the editor.
\item \LaTeX{} 소스파일을 편집하기위한 텍스트 편집기.
어떤 프로그램은 \LaTeX{} 프로그램을 편집기 안에서 시작하기도 한다.

%\item A PDF/DVI viewer program for previewing and printing your
%documents.
\item A PDF/DVI viewer program for previewing and printing your
documents.

%\item A program to handle \PSi{} files and images for inclusion into
%your documents.
\item A program to handle \PSi{} files and images for inclusion into
your documents.

\end{enumerate}

%For every platforms there are several programs that fit the requirements above.
%Here we just tell about the ones we know, like and have some experience
%with.
For every platforms there are several programs that fit the requirements above.
Here we just tell about the ones we know, like and have some experience
with.

%\section{Cross Platform Editor}
\section{크로스 플랫폼 편집기}
%\label{sec:texmaker}
\label{sec:texmaker}

%While \TeX{} is available on many different computing platforms, \LaTeX{}
%editors have long been highly platform specific.
\TeX{}은 여러 다른 컴퓨터에서 사용할 수 있지만, \LaTeX{} 편집기는 플랫폼을 매우 많이 탄다.

%Over the past few years I have come to like Texmaker quite a lot.
%Apart from being very a useful editor with integrated pdf-preview and syntax
%high-lighting, it has the advantage of running on Windows, Mac and
%Unix/Linux equally well.  See \url{http://www.xm1math.net/texmaker} for
%further information.  There is also a forked version of Texmaker called
%TeXstudio on \url{http://texstudio.sourceforge.net/}.  It also seems well
%maintained and is also available for all three major platforms.
Over the past few years I have come to like Texmaker quite a lot.
Apart from being very a useful editor with integrated pdf-preview and syntax
high-lighting, it has the advantage of running on Windows, Mac and
Unix/Linux equally well.  See \url{http://www.xm1math.net/texmaker} for
further information.  There is also a forked version of Texmaker called
TeXstudio on \url{http://texstudio.sourceforge.net/}.  It also seems well
maintained and is also available for all three major platforms.

%You will find some platform specific editor suggestions in the OS sections
%below.
아래의 OS절에서 플랫폼에 따른 편집기에 대한 제안을 찾을 수 있을 것이다.

\section{\TeX{}을 Mac OS X에 설치}

%\subsection{\TeX{} Distribution}
\subsection{\TeX{} 배포}

Just download \wi{MacTeX}. It is a
pre-compiled \LaTeX{} distribution for OS X. \wi{MacTeX} provides a full \LaTeX{}
installation plus a number of additional tools. Get Mac\TeX{} from
\url{http://www.tug.org/mactex/}.

%\subsection{OSX \TeX{} Editor}
\subsection{OSX \TeX{} 편집기}

%If you are not happy with our crossplatform suggestion Texmaker (section \ref{sec:texmaker}).
If you are not happy with our crossplatform suggestion Texmaker (section \ref{sec:texmaker}).
 
The most popular open source editor for \LaTeX{} on the mac seems to be
\TeX{}shop.  Get a copy from \url{http://www.uoregon.edu/~koch/texshop}. It
is also contained in the \wi{MacTeX} distribution.

%Recent \TeX Live distributions contain the \TeX{}works editor 
%\url{http://texworks.org/} which is a multi-platform editor based on the \TeX{}Shop
%design. Since \TeX{}works uses the Qt toolkit, it is available on any platform
%supported by this toolkit (MacOS X, Windows, Linux.) 
Recent \TeX Live distributions contain the \TeX{}works editor 
\url{http://texworks.org/} which is a multi-platform editor based on the \TeX{}Shop
design. Since \TeX{}works uses the Qt toolkit, it is available on any platform
supported by this toolkit (MacOS X, Windows, Linux.) 

%\subsection{Treat yourself to \wi{PDFView}}
\subsection{Treat yourself to \wi{PDFView}}

Use PDFView for viewing PDF files generated by \LaTeX{}, it integrates tightly
with your \LaTeX{} text editor. PDFView is an open-source application, available from the PDFView website on\\
\url{http://pdfview.sourceforge.net/}. After installing, open
PDFViews preferences dialog and make sure that the \emph{automatically reload
documents} option is enabled and that PDFSync support is set appropriately.

%\section{\TeX{} on Windows}
\section{\TeX{}을 윈도에 설치}

%\subsection{Getting \TeX{}}
\subsection{\TeX{} 얻기}
%First, get a copy of the excellent MiK\TeX\index{MiKTeX@MiK\TeX} distribution from\\
%\url{http://www.miktex.org/}. It contains all the basic programs and files
%required to compile \LaTeX{} documents.  The coolest feature in my eyes, is
%that MiK\TeX{} will download missing \LaTeX{} packages on the fly and install them
%magically while compiling a document. Alternatively you can also use
%the TeXlive distribution which exists for Windows, Unix and Mac OS to
%get your base setup going \url{http://www.tug.org/texlive/}.
First, get a copy of the excellent MiK\TeX\index{MiKTeX@MiK\TeX} distribution from\\
\url{http://www.miktex.org/}. It contains all the basic programs and files
required to compile \LaTeX{} documents.  The coolest feature in my eyes, is
that MiK\TeX{} will download missing \LaTeX{} packages on the fly and install them
magically while compiling a document. Alternatively you can also use
the TeXlive distribution which exists for Windows, Unix and Mac OS to
get your base setup going \url{http://www.tug.org/texlive/}.

%\subsection{A \LaTeX{} editor}
\subsection{A \LaTeX{} 편집기}
%If you are not happy with our crossplatform suggestion Texmaker (section \ref{sec:texmaker}).
If you are not happy with our crossplatform suggestion Texmaker (section \ref{sec:texmaker}).

%\wi{TeXnicCenter} uses many concepts from the programming-world to provide a nice and
%efficient \LaTeX{} writing environment in Windows. Get your copy from\\
%\url{http://www.texniccenter.org/}. TeXnicCenter integrates nicely with
%MiKTeX.
\wi{TeXnicCenter} uses many concepts from the programming-world to provide a nice and
efficient \LaTeX{} writing environment in Windows. Get your copy from\\
\url{http://www.texniccenter.org/}. TeXnicCenter integrates nicely with
MiKTeX.

%Recent \TeX Live distributions contain the \TeX{}works Editor
%\url{http://texworks.org/}. It supports Unicode and requires at least Windows XP.
Recent \TeX Live distributions contain the \TeX{}works Editor
\url{http://texworks.org/}. It supports Unicode and requires at least Windows XP.

%\subsection{Document Preview}
\subsection{문서 미리보기}
%You will most likely be using Yap for DVI preview as it gets installed with
%MikTeX. For PDF you may want to look at Sumatra
%PDF \url{http://blog.kowalczyk.info/software/sumatrapdf/}. I mention Sumatra PDF
%because it lets you jump from any position in the pdf document back into
%corresponding position in your source document.
You will most likely be using Yap for DVI preview as it gets installed with
MikTeX. For PDF you may want to look at Sumatra
PDF \url{http://blog.kowalczyk.info/software/sumatrapdf/}. I mention Sumatra PDF
because it lets you jump from any position in the pdf document back into
corresponding position in your source document.

%\subsection{Working with graphics}
\subsection{그래픽 작업하기}
%Working with high quality graphics in \LaTeX{} means that you have to use
%\EPSi{} (eps) or PDF as your picture format. The program that helps you
%deal with this is called \wi{GhostScript}. You can get it, together with its
%own front-end \wi{GhostView}, from \url{http://www.cs.wisc.edu/~ghost/}.
Working with high quality graphics in \LaTeX{} means that you have to use
\EPSi{} (eps) or PDF as your picture format. The program that helps you
deal with this is called \wi{GhostScript}. You can get it, together with its
own front-end \wi{GhostView}, from \url{http://www.cs.wisc.edu/~ghost/}.

%If you deal with bitmap graphics (photos and scanned material), you may want
%to have a look at the open source Photoshop alternative \wi{Gimp}, available
%from \url{http://gimp-win.sourceforge.net/}.
비트맵 그래펵 (사진과 스캔한 재료)을 다룬다면, 
오픈소스 포토샵 대안인 \wi{Gimp}를 원할 수도 있는데, 
그것은 \url{http://gimp-win.sourceforge.net/}에서 받아서 사용 가능하다.

%\section{\TeX{} on Linux}
\section{\TeX{}을 리눅스에 설치}

%If you work with Linux, chances are high that \LaTeX{} is already installed
%on your system, or at least available on the installation source you used to
%setup. Use your package manager to install the following packages:
리눅스로 작업한다면, \LaTeX{}가 여러분의 시스템에 이미 설치되어 있을 가능성이 높다.
또는 적어도 여러분이 쓰던 컴퓨터에 소스를 설치하는 것이 가능하다.
아래 패키지를 설치하기 위해 패키지 관리자를 사용하라.

\begin{itemize}
\item texlive -- 기본 \TeX{}/\LaTeX{} 설치.
\item emacs (AUCTeX 포함) -- an editor that integrates tightly with \LaTeX{} through the add-on AUCTeX package.
\item ghostscript --  \PSi{} 미리보기 프로그램.
\item xpdf와 acrobat -- PDF 미리보기 프로그램
\item imagemagick -- 비트맵 이미지를 변환하는 무료 프로그램
\item gimp -- 포토샵  비슷한 무료 프로그램
\item inkscape -- illustrator/corel draw 비슷한 무료 프로그램
\end{itemize}

%If you are looking for a more windows like graphical editing environment,
%check out Texmaker. See section \ref{sec:texmaker}.
좀 더 windows 비슷한 그래픽 편집 환경을 찾는다면,
Texmaker를 살펴보라. \ref{sec:texmaker}를 보시오.

%Most Linux distros insist on splitting up their \TeX{} environments into a
%large number of optional packages, so if something is missing after your
%first install, go check again.
대부분의 리눅스 배포본은  \TeX{} 환경을 
여러 개의 선택적 패키지로 나누는 것을 요구하므로, 처음 설치후 무언가 잘못되면,
다시 한 번 살펴보시오.
